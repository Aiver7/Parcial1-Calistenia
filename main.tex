\documentclass{article}
\usepackage[utf8]{inputenc}
\usepackage[spanish]{babel}
\usepackage{listings}
\usepackage{graphicx}
\graphicspath{ {images/} }
\usepackage{cite}

\begin{document}

\begin{titlepage}
    \begin{center}
        \vspace*{1cm}
            
        \Huge
        \textbf{Parcial 1 - Calistenia}
            
        \vspace{0.5cm}
        \LARGE
        
            
        \vspace{1.5cm}
            
        \textbf{Alex Aiverson Palacios Mosquera}
            
        \vfill
            
        \vspace{0.8cm}
            
        \Large
        Despartamento de Ingeniería Electrónica y Telecomunicaciones\\
        Universidad de Antioquia\\
        Medellín\\
        Marzo de 2021
            
    \end{center}
\end{titlepage}


\section{Introducción}\label{intro}
En este documento veremos una serie de pasos para lograr mantener 2 tarjetas paradas en forma de piramide sobre un a hoja de bond en una superficie plana.

\section{Instrucciones} \label{contenido}
A continuación veremos las instrucciones:
\\
1) Ponga las tarjetas sobre la superficie
\\
2) Ponga la hoja sobre las tarjetas
\\
3) Ponga el dedo indice de la mano derecha sobre un extremo de la hoja
\\
4) Coja las tarjetas
\\
5) Levantelas
\\
4) Pongalas en posicion vertical
\\
5) Separelas hasta formar una piramide
\\
6) Mantenga el equilibrio
\\
7) Suelte



\bibliographystyle{IEEEtran}
\bibliography{references}

\end{document}
